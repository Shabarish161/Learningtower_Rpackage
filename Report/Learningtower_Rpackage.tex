% Options for packages loaded elsewhere
\PassOptionsToPackage{unicode}{hyperref}
\PassOptionsToPackage{hyphens}{url}
\PassOptionsToPackage{dvipsnames,svgnames,x11names}{xcolor}
%
\documentclass[
  11pt,
  a4paper,
]{article}

\usepackage{amsmath,amssymb}
\usepackage{setspace}
\usepackage{iftex}
\ifPDFTeX
  \usepackage[T1]{fontenc}
  \usepackage[utf8]{inputenc}
  \usepackage{textcomp} % provide euro and other symbols
\else % if luatex or xetex
  \usepackage{unicode-math}
  \defaultfontfeatures{Scale=MatchLowercase}
  \defaultfontfeatures[\rmfamily]{Ligatures=TeX,Scale=1}
\fi
\usepackage{lmodern}
\ifPDFTeX\else  
    % xetex/luatex font selection
\fi
% Use upquote if available, for straight quotes in verbatim environments
\IfFileExists{upquote.sty}{\usepackage{upquote}}{}
\IfFileExists{microtype.sty}{% use microtype if available
  \usepackage[]{microtype}
  \UseMicrotypeSet[protrusion]{basicmath} % disable protrusion for tt fonts
}{}
\makeatletter
\@ifundefined{KOMAClassName}{% if non-KOMA class
  \IfFileExists{parskip.sty}{%
    \usepackage{parskip}
  }{% else
    \setlength{\parindent}{0pt}
    \setlength{\parskip}{6pt plus 2pt minus 1pt}}
}{% if KOMA class
  \KOMAoptions{parskip=half}}
\makeatother
\usepackage{xcolor}
\usepackage[top=2.4cm,bottom=2.4cm,left=2.5cm,right=2.5cm]{geometry}
\setlength{\emergencystretch}{3em} % prevent overfull lines
\setcounter{secnumdepth}{2}


\providecommand{\tightlist}{%
  \setlength{\itemsep}{0pt}\setlength{\parskip}{0pt}}\usepackage{longtable,booktabs,array}
\usepackage{calc} % for calculating minipage widths
% Correct order of tables after \paragraph or \subparagraph
\usepackage{etoolbox}
\makeatletter
\patchcmd\longtable{\par}{\if@noskipsec\mbox{}\fi\par}{}{}
\makeatother
% Allow footnotes in longtable head/foot
\IfFileExists{footnotehyper.sty}{\usepackage{footnotehyper}}{\usepackage{footnote}}
\makesavenoteenv{longtable}
\usepackage{graphicx}
\makeatletter
\def\maxwidth{\ifdim\Gin@nat@width>\linewidth\linewidth\else\Gin@nat@width\fi}
\def\maxheight{\ifdim\Gin@nat@height>\textheight\textheight\else\Gin@nat@height\fi}
\makeatother
% Scale images if necessary, so that they will not overflow the page
% margins by default, and it is still possible to overwrite the defaults
% using explicit options in \includegraphics[width, height, ...]{}
\setkeys{Gin}{width=\maxwidth,height=\maxheight,keepaspectratio}
% Set default figure placement to htbp
\makeatletter
\def\fps@figure{htbp}
\makeatother

\makeatletter
\@ifpackageloaded{caption}{}{\usepackage{caption}}
\AtBeginDocument{%
\ifdefined\contentsname
  \renewcommand*\contentsname{Table of contents}
\else
  \newcommand\contentsname{Table of contents}
\fi
\ifdefined\listfigurename
  \renewcommand*\listfigurename{List of Figures}
\else
  \newcommand\listfigurename{List of Figures}
\fi
\ifdefined\listtablename
  \renewcommand*\listtablename{List of Tables}
\else
  \newcommand\listtablename{List of Tables}
\fi
\ifdefined\figurename
  \renewcommand*\figurename{Figure}
\else
  \newcommand\figurename{Figure}
\fi
\ifdefined\tablename
  \renewcommand*\tablename{Table}
\else
  \newcommand\tablename{Table}
\fi
}
\@ifpackageloaded{float}{}{\usepackage{float}}
\floatstyle{ruled}
\@ifundefined{c@chapter}{\newfloat{codelisting}{h}{lop}}{\newfloat{codelisting}{h}{lop}[chapter]}
\floatname{codelisting}{Listing}
\newcommand*\listoflistings{\listof{codelisting}{List of Listings}}
\makeatother
\makeatletter
\makeatother
\makeatletter
\@ifpackageloaded{caption}{}{\usepackage{caption}}
\@ifpackageloaded{subcaption}{}{\usepackage{subcaption}}
\makeatother
\ifLuaTeX
  \usepackage{selnolig}  % disable illegal ligatures
\fi
\usepackage[style=authoryear-comp,]{biblatex}
\addbibresource{references.bib}
\usepackage{bookmark}

\IfFileExists{xurl.sty}{\usepackage{xurl}}{} % add URL line breaks if available
\urlstyle{same} % disable monospaced font for URLs
\hypersetup{
  pdftitle={Report of developing Learningtower R package},
  pdfauthor={Shabarish Sai Subramanian; Guan Ru, Chen},
  colorlinks=true,
  linkcolor={blue},
  filecolor={Maroon},
  citecolor={Blue},
  urlcolor={Blue},
  pdfcreator={LaTeX via pandoc}}

%% CAPTIONS
\usepackage{caption}
\DeclareCaptionStyle{italic}[justification=centering]
 {labelfont={bf},textfont={it},labelsep=colon}
\captionsetup[figure]{style=italic,format=hang,singlelinecheck=true}
\captionsetup[table]{style=italic,format=hang,singlelinecheck=true}

%% FONT
\usepackage{bera}
\usepackage[charter]{mathdesign}
\usepackage[scale=0.9]{sourcecodepro}
\usepackage[lf,t]{FiraSans}
\usepackage{fontawesome}

%% HEADERS AND FOOTERS
\usepackage{fancyhdr}
\pagestyle{fancy}
\rfoot{\Large\sffamily\raisebox{-0.1cm}{\textbf{\thepage}}}
\makeatletter
\lhead{\textsf{\expandafter{\@title}}}
\makeatother
\rhead{}
\cfoot{}
\setlength{\headheight}{15pt}
\renewcommand{\headrulewidth}{0.4pt}
\renewcommand{\footrulewidth}{0.4pt}
\fancypagestyle{plain}{%
\fancyhf{} % clear all header and footer fields
\fancyfoot[C]{\sffamily\thepage} % except the center
\renewcommand{\headrulewidth}{0pt}
\renewcommand{\footrulewidth}{0pt}}

%% MATHS
\usepackage{bm,amsmath}
\allowdisplaybreaks

%% GRAPHICS
\makeatletter
\def\fps@figure{htbp}
\makeatother
\setcounter{topnumber}{2}
\setcounter{bottomnumber}{2}
\setcounter{totalnumber}{4}
\renewcommand{\topfraction}{0.85}
\renewcommand{\bottomfraction}{0.85}
\renewcommand{\textfraction}{0.15}
\renewcommand{\floatpagefraction}{0.8}

%% SECTION TITLES
\usepackage[compact,sf,bf]{titlesec}
\titleformat*{\section}{\Large\sf\bfseries\color[rgb]{0.7,0,0}}
\titleformat*{\subsection}{\large\sf\bfseries\color[rgb]{0.7,0,0}}
\titleformat*{\subsubsection}{\sf\bfseries\color[rgb]{0.7,0,0}}
\titlespacing{\section}{0pt}{*5}{*1}
\titlespacing{\subsection}{0pt}{*2}{*0.2}
\titlespacing{\subsubsection}{0pt}{*1}{*0.1}

%% BIBLIOGRAPHY.

\makeatletter
\@ifpackageloaded{biblatex}{
\ExecuteBibliographyOptions{bibencoding=utf8,minnames=1,maxnames=3, maxbibnames=99,dashed=false,terseinits=true,giveninits=true,uniquename=false,uniquelist=false,doi=false, isbn=false,url=true,sortcites=false}
\DeclareFieldFormat{url}{\texttt{\url{#1}}}
\DeclareFieldFormat[article]{pages}{#1}
\DeclareFieldFormat[inproceedings]{pages}{\lowercase{pp.}#1}
\DeclareFieldFormat[incollection]{pages}{\lowercase{pp.}#1}
\DeclareFieldFormat[article]{volume}{\mkbibbold{#1}}
\DeclareFieldFormat[article]{number}{\mkbibparens{#1}}
\DeclareFieldFormat[article]{title}{\MakeCapital{#1}}
\DeclareFieldFormat[article]{url}{}
\DeclareFieldFormat[inproceedings]{title}{#1}
\DeclareFieldFormat{shorthandwidth}{#1}
\usepackage{xpatch}
\xpatchbibmacro{volume+number+eid}{\setunit*{\adddot}}{}{}{}
% Remove In: for an article.
\renewbibmacro{in:}{%
  \ifentrytype{article}{}{%
  \printtext{\bibstring{in}\intitlepunct}}}
\AtEveryBibitem{\clearfield{month}}
\AtEveryCitekey{\clearfield{month}}
\DeclareDelimFormat[cbx@textcite]{nameyeardelim}{\addspace}
\renewcommand*{\finalnamedelim}{\addspace\&\space}
}{}
\makeatother

%% PAGE BREAKING to avoid widows and orphans
\clubpenalty = 2000
\widowpenalty = 2000
\usepackage{microtype}% Placement of logos

\RequirePackage[absolute,overlay]{textpos}
\setlength{\TPHorizModule}{1cm}
\setlength{\TPVertModule}{1cm}
\def\placefig#1#2#3#4{\begin{textblock}{.1}(#1,#2)\rlap{\includegraphics[#3]{#4}}\end{textblock}}

% Title and date

\title{Report of developing Learningtower R package}
\date{9 October 2024}

\def\Date{\number\day}
\def\Month{\ifcase\month\or
 January\or February\or March\or April\or May\or June\or
 July\or August\or September\or October\or November\or December\fi}
\def\Year{\number\year}

%%%% PAGE STYLE FOR FRONT PAGE OF REPORTS

\makeatletter
\def\organization#1{\gdef\@organization{#1}}
\def\telephone#1{\gdef\@telephone{#1}}
\def\email#1{\gdef\@email{#1}}
\makeatother
  \organization{Department of Econometrics \& Business Statistics,
Monash University}

  \def\name{Department of\newline Econometrics \&\newline Business Statistics}

  \telephone{(03) 9905 2478}

  \email{BusEco-Econometrics@monash.edu}

\def\webaddress{\url{https://www.monash.edu/business/ebs}}
\def\abn{12 377 614 012}
\def\extraspace{\vspace*{1.6cm}}
\makeatletter
\def\contactdetails{\faicon{phone} & \@telephone \\
                    \faicon{envelope} & \@email}
\makeatother

\usepackage[absolute,overlay]{textpos}
\setlength{\TPHorizModule}{1cm}
\setlength{\TPVertModule}{1cm}

%%%% FRONT PAGE OF REPORTS

\def\reporttype{Report for}

\long\def\front#1#2#3{
\newpage
\begin{textblock}{7}(12.7,28.2)\hfill
\includegraphics[height=0.6cm]{AACSB}~~~
\includegraphics[height=0.6cm]{EQUIS}~~~
\includegraphics[height=0.6cm]{AMBA}
\end{textblock}
\begin{singlespacing}
\thispagestyle{empty}
\vspace*{-1.4cm}
\hspace*{-1.4cm}
\hbox to 16cm{
  \hbox to 6.5cm{\vbox to 14cm{\vbox to 25cm{
    \includegraphics[width=6cm]{monash2}
    \vfill
    \includegraphics[width=2cm]{MBSportrait}
    \vspace{0.4cm}
    \par
    \parbox{6.3cm}{\raggedright
      \sf\color[rgb]{0.00,0.00,0.70}
      {\large\textbf{\name}}\par
      \vspace{.7cm}
      \tabcolsep=0.12cm\sf\small
      \begin{tabular}{@{}ll@{}}\contactdetails
      \end{tabular}
      \vspace*{0.3cm}\par
      ABN: \abn\par
    }
  }\vss}\hss}
  \hspace*{0.2cm}
  \hbox to 1cm{\vbox to 14cm{\rule{1pt}{26.8cm}\vss}\hss\hfill}
  \hbox to 10cm{\vbox to 14cm{\vbox to 25cm{
      \vspace*{3cm}\sf\raggedright
      \parbox{10cm}{\sf\raggedright\baselineskip=1.2cm
         \fontsize{24.88}{30}\color[rgb]{0.70,0.00,0.00}\sf\textbf{#1}}
      \par
      \vfill
      \large
      \vbox{\parskip=0.8cm #2}\par
      \vspace*{2cm}\par
      \reporttype\\[0.3cm]
      \hbox{#3}%\\[2cm]\
      \vspace*{1cm}
      {\large\sf\textbf{\Date~\Month~\Year}}
   }\vss}
  }}
\end{singlespacing}
\newpage
}

\makeatletter
\def\maketitle{\front{\expandafter{\@title}}{\@author}{\@organization}}
\makeatother

% Authors

\author{\sf{\Large\textbf{Shabarish Sai Subramanian} \\\large Master of
Businss Analytics\\[0.5cm]}{\Large\textbf{Guan Ru, Chen} \\\large Master
of Businss Analytics\\[0.5cm]}}
\lfoot{\sf Subramanian, Guan Ru: 9 October 2024}
\begin{document}
\maketitle

\setstretch{1.5}
\subsection{Background}\label{background}

The learningtower package, which focusses on school and student-level
data like performance indicators, socioeconomic backgrounds, and
educational resources, contains educational datasets from international
exams like PISA before 2022. After being cleaned and standardised, these
datasets go through modifications including variable alignment for
consistency and student-teacher ratio calculations. Key performance
indicators include student performance in disciplines including reading,
mathematics, and science; school and country identities; and educational
resources (e.g., staff shortages, school size). The information makes it
possible to analyse regional and worldwide trends as well as differences
in educational systems, which sheds light on the variables influencing
resource allocation and academic performance.

\subsection{Introduction}\label{introduction}

\subsection{PISA}\label{pisa}

The Organization for Economic Cooperation and Development
\href{https://www.oecd.org/about/}{OECD} is a global organization that
aims to create better policies for better lives. Its mission is to
create policies that promote prosperity, equality, opportunity, and
well-being for all. \autocite{oecd}
\href{https://www.oecd.org/pisa/}{PISA} is one of OECD's Programme for
International Student Assessment. PISA assesses 15-year-old students'
potential to apply their knowledge and abilities in reading,
mathematics, and science to real-world challenges. OECD launched this in
1997, it was initially administered in 2000, and it currently includes
over
\href{https://www.oecd.org/pisa/aboutpisa/pisa-participants.html}{80
nations}. \autocite{pisa} The PISA study, conducted every three years,
provides comparative statistics on 15-year-old students' performance in
reading, math, and science. This report describes how to utilize the
\texttt{learningtower} package, which offers OECD PISA datasets from
2000 to 2022 in an easy-to-use format. The datasets comprise information
on students' test results and other socioeconomic factors, as well as
information on their schools, infrastructure and the countries
participating in the program.

\subsection{Learningtower Package}\label{learningtower-package}

\href{https://cran.r-project.org/web/packages/learningtower/index.html}{`learningtower'}
The R package \autocite{learningtower} provides quick access to a
variety of variables in the OECD PISA data collected over a three-year
period from 2000 to 2022. This dataset includes information on the PISA
test scores in mathematics, reading, and science. Furthermore, these
datasets include information on other socioeconomic aspects, as well as
information on their school and its facilities, as well as the nations
participating in the program.

The \texttt{learningtower} package primarily comprised of three
datasets: \texttt{student}, \texttt{school}, and \texttt{countrycode.}
The \texttt{student} dataset includes results from triennial testing of
15-year-old students throughout the world. This dataset also includes
information about their parents' education, family wealth, gender, and
presence of computers, internet, vehicles, books, rooms, desks, and
other comparable factors. Due to the size limitation on CRAN packages,
only a subset of the student data can be made available in the
downloaded package. These subsets of the student data, known as the
\texttt{student\_subset\_yyyy} (\texttt{yyyy} being the specific year of
the study) allow uses to quickly load, visualise the trends in the full
data. The full student dataset can be downloaded using the
\texttt{load\_student()} function included in this
\href{https://kevinwang09.github.io/learningtower/}{package.} The
\texttt{school} dataset includes school weight as well as other
information such as school funding distribution, whether the school is
private or public, enrollment of boys and girls, school size, and
similar other characteristics of interest of different schools these
15-year-olds attend around the world. The \texttt{countrycode} dataset
includes a mapping of a country/region's ISO code to its full name.

\subsection{Goals}\label{goals}

The motivation for developing the \texttt{learningtower} package was
sparked by the announcement of the PISA 2018 results, which caused a
collective wringing of hands in the Australian press, with headlines
such as
\href{https://theconversation.com/vital-signs-australias-slipping-student-scores-will-lead-to-greater-income-inequality-128301}{``Vital
Signs: Australia's slipping student scores will lead to greater income
inequality''} and
\href{https://www.smh.com.au/education/in-china-nicholas-studied-maths-20-hours-a-week-in-australia-it-s-three-20191203-p53ggv.html}{``In
China, Nicholas studied math 20 hours a week. In Australia, it's
three''}. That's when several academics from Australia, New Zealand, and
Indonesia decided to make things easier by providing easy access to PISA
scores as part of the \href{https://ozunconf19.ropensci.org/}{ROpenSci
OzUnconf}, which was held in Sydney from December 11 to 13, 2019.

The data from this survey, as well as all other surveys performed since
the initial collection in 2000, is freely accessible to the public.
However, downloading and curating data across multiple years of the PISA
study could be a time consuming task. As a result, we have made a more
convenient subset of the data freely available in a new R package called
\texttt{learningtower}, along with sample code for analysis.

\texttt{learningtower} developers are committed to providing R users
with data to analyse PISA results every three years. Our package's
future enhancements include updating the package every time additional
PISA scores are announced. Note that, in order to account for post
COVID-19 problems, OECD member nations and associates decided to
postpone the PISA 2021 evaluation to 2022 and the PISA 2024 assessment
to 2025.

\subsection{Compiling the Data(more details about the process and
problems
faced)}\label{compiling-the-datamore-details-about-the-process-and-problems-faced}

We are responsible for the curation of the newest PISA study, year 2022.
Data on the participating students and schools were first downloaded
from the PISA website, in either SPSS or SAS format. The data were read
into an R environment. After some data cleaning and wrangling with the
appropriate script, the variables of interest were re-categorised and
saved as RDS files. One major challenge faced by the us was to ensure
the consistency of variables over the years. However, several variables
may be missing due to the reconstruction of questionnaires. For
instance, a question regarding student's possession of desk is not
recorded in 2022, but it was included in previous questionnaires, hence
these variables were manually curated as an character variable in the
output data. Another important issue we faced is a missing variable
\texttt{WEALTH}, this variable used to be a good measurement of a
student's socioeconomic status. But we also discovered a variable called
\texttt{ESCS} (economic, social and cultural status). Also for 2022 PISA
test, the student's curiosity was the first time included in the
questionnaire. As
\href{https://www.teachermagazine.com/au_en/articles/pisa-2022-student-curiosity-helps-maths-performance}{``the
finding show a clear link between student curiosity and student
performance in mathematics.''}, we decide to includ this attribute in
the student dataset as well. These final RDS file for each PISA year
were then thoroughly vetted and made available in a separate
\href{https://github.com/kevinwang09/learningtower_masonry}{GitHub
repository}.

\subsection{Communication and Documentation
Tools}\label{communication-and-documentation-tools}

Slack and Notion can be effectively utilized together to enhance team
communication and documentation management. Slack serves as a real-time
communication tool, allowing teams to quickly exchange information,
discuss projects, and stay updated on tasks, making it ideal for team
collaboration.

Notion, on the other hand, excels as a centralized workspace for
recording and organizing important documents, such as meeting journals,
project notes, and other key materials, ensuring that important
information is organized and easily accessible.

By using Slack for dynamic conversations and Notion for structured
documentation, teams can ensure seamless communication while maintaining
an organized record of all important documents, meeting notes, and
long-term planning.

\section{Overview of the data}\label{overview-of-the-data}

\section{Analysis}\label{analysis}

\subsubsection{Gender}\label{gender}

\subsubsection{Pandemic effects}\label{pandemic-effects}

\subsubsection{EcoSocio factors}\label{ecosocio-factors}

\section{Comparison with previous years'
datasets}\label{comparison-with-previous-years-datasets}

\section{Reference}\label{reference}

\subsection{Git respository of the
report}\label{git-respository-of-the-report}

https://github.com/Shabarish161/Learningtower\_Rpackage


\printbibliography


\end{document}
